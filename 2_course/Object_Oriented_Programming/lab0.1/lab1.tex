\documentclass[12pt]{article}

\usepackage{fullpage}
\usepackage{multicol,multirow}
\usepackage{tabularx}
\usepackage{ulem}
\usepackage[utf8]{inputenc}
\usepackage[russian]{babel}
\usepackage{minted}

\usepackage{color} %% это для отображения цвета в коде
\usepackage{listings} %% собственно, это и есть пакет listings

\lstset{ %
language=C++,                 % выбор языка для подсветки (здесь это С++)
basicstyle=\small\sffamily, % размер и начертание шрифта для подсветки кода
numbers=left,               % где поставить нумерацию строк (слева\справа)
%numberstyle=\tiny,           % размер шрифта для номеров строк
stepnumber=1,                   % размер шага между двумя номерами строк
numbersep=5pt,                % как далеко отстоят номера строк от подсвечиваемого кода
backgroundcolor=\color{white}, % цвет фона подсветки - используем \usepackage{color}
showspaces=false,            % показывать или нет пробелы специальными отступами
showstringspaces=false,      % показывать или нет пробелы в строках
showtabs=false,             % показывать или нет табуляцию в строках
frame=single,              % рисовать рамку вокруг кода
tabsize=2,                 % размер табуляции по умолчанию равен 2 пробелам
captionpos=t,              % позиция заголовка вверху [t] или внизу [b] 
breaklines=true,           % автоматически переносить строки (да\нет)
breakatwhitespace=false, % переносить строки только если есть пробел
escapeinside={\%*}{*)}   % если нужно добавить комментарии в коде
}


\begin{document}
\begin{titlepage}
\begin{center}
\textbf{МИНИСТЕРСТВО ОБРАЗОВАНИЯ И НАУКИ РОССИЙСКОЙ ФЕДЕРАЦИИ
\medskip
МОСКОВСКИЙ АВИАЦИОННЫЙ ИНСТИТУТ
(НАЦИОНАЛЬНЫЙ ИССЛЕДОВАТЕЛЬСКИЙ УНИВЕРСТИТЕТ)
\vfill\vfill
{\Huge ЛАБОРАТОРНАЯ РАБОТА №1} \\
по курсу объектно-ориентированное программирование
I семестр, 2021/22 уч. год}
\end{center}
\vfill

Студент \uline{\it {Каширин Кирилл Дмитриевич, группа М8О-208Б-20}\hfill}

Преподаватель \uline{\it {Дорохов Евгений Павлович}\hfill}

\vfill
\end{titlepage}

\subsection*{Условие}
Создать класс BitString для работы с 128-битовыми строками. Битовая строка
должна быть представлена двумя полями типа unsigned long long. Должны быть
реализованы все традиционные операции для работы с битами: and, or, xor, not.
Реализовать сдвиг влево shiftLeft и сдвиг вправо shiftRight на заданное количество битов. Реализовать операцию вычисления количества единичных битов, операции сравнения по количеству единичных битов. Реализовать операцию проверки
включения.
Исходный код лежит в трёх файлах:
\begin{enumerate}
\item main.cpp: основная программа, взаимодействие с пользователем посредством команд из меню
\item BitString.h:    описание класса адресов
\item BitString.cpp:  реализация класса адреса

\end{enumerate}
\pagebreak
\subsection*{Протокол работы}
12345 6789 \\
BitString a: \\
00000000000000000000000000000000000000000000000001011011100000010000000000000000000000000000000000000000000000000111111010011101 \\
BitString b: \\
00000000000000000000000000000000000000000000000000110000001110010000000000000000000000000000000000000000000000000001101010000101 \\
AND \\
00000000000000000000000000000000000000000000000000010000000000010000000000000000000000000000000000000000000000000001101010000101 \\
OR \\
00000000000000000000000000000000000000000000000001111011101110010000000000000000000000000000000000000000000000000111111010011101 \\
XOR \\
00000000000000000000000000000000000000000000000001101011101110000000000000000000000000000000000000000000000000000110010000011000 \\
NOT \\
11111111111111111111111111111111111111111111111110100100011111101111111111111111111111111111111111111111111111111000000101100010 \\
1 in a \\
18 \\
0 \\
0 \\
\subsection*{Дневник отладки}
Проблем и ошибок при написании данной работы не возникло.

\subsection*{Недочёты}


\subsection*{Выводы}
В процессе выполнения работы я на практике познакомился с классами. Эта лабораторная является вводной по изучению объекто-ориентированной парадигмы. Классы удобны для упрощения написание кода для различных объемных программ, ипользующих различные типы данных, содержащие сразу несколько различных полей. Например, при необходимости можно реализовать новый тип данных - битовая строка, которая работает с двумя полями unsigned long long.


\vfill
\pagebreak
\subsection*{Исходный код:}

{\Huge BitString.h}
\inputminted
    {C++}{BitString.h}
    
{\Huge BitString.cpp}
\inputminted
    {C++}{BitString.cpp}
    
{\Huge main.cpp}
\inputminted
    {C++}{main.cpp}
    
\end{document}